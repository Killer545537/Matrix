\documentclass[titlepage]{article}
\usepackage[margin=0.5in]{geometry}
\usepackage{amsmath}
\usepackage{amsfonts}
\usepackage{amssymb}
\usepackage{amsthm}
\usepackage{graphicx}
\usepackage{physics}
\usepackage{tikz}
\usepackage{pgfplots}
\pgfplotsset{compat=1.15}
\usepackage{mathrsfs}
\usetikzlibrary{arrows,patterns, backgrounds, calc,fadings,shadows.blur, shapes}
\theoremstyle{plain}
\usepackage[most]{tcolorbox}
\usepackage{physunits}
\usepackage{float}
\usepackage{xcolor}
\usepackage{microtype}
\usepackage{thmtools}
\usepackage[framemethod=TikZ]{mdframed}
\usepackage{steinmetz}
\usepackage{adjustbox}
\usepackage{listings}
\mdfsetup{skipabove=1em,skipbelow=0em}
\tcbuselibrary{xparse}
\usepackage[font=small,labelfont=bf,margin=\parindent,tableposition=top]{caption}
\newenvironment{solution}
{\renewcommand\qedsymbol{$\blacksquare$}\begin{proof}[Solution]}
	{\end{proof}}


\declaretheoremstyle[
headfont=\bfseries\sffamily\color{blue!70!black}, bodyfont=\normalfont,
mdframed={
	linewidth=1pt,
	rightline=false, topline=false, bottomline=false,
	linecolor=blue, backgroundcolor=blue!5,
}
]{thmbluebox}
\declaretheorem[style=thmbluebox, numbered=no, name=Problem]{eg}


\declaretheoremstyle[
headfont=\bfseries\sffamily\color{blue!70!black}, bodyfont=\normalfont,
numbered=no,
mdframed={
	linewidth=1pt,
	rightline=false, topline=false, bottomline=false,
	linecolor=blue, backgroundcolor=blue!1,
},
]{thmexplanationbox}
\declaretheorem[style=thmexplanationbox, name=Code]{tmpexplanation}
\newenvironment{explanation}[1][]{\vspace{-10pt}\begin{tmpexplanation}}{\end{tmpexplanation}}


\declaretheoremstyle[
headfont=\bfseries\sffamily\color{red!70!black}, bodyfont=\normalfont,
mdframed={
	linewidth=1pt,
	rightline=false, topline=false, bottomline=false,
	linecolor=red, backgroundcolor=red!5,
}
]{thmredbox}
\declaretheorem[style=thmredbox,numbered=no, name=\phantom{.}]{theorem}

\declaretheoremstyle[
headfont=\bfseries\sffamily\color{red!70!black}, bodyfont=\normalfont,
numbered=no,
mdframed={
	linewidth=1pt,
	rightline=false, topline=false, bottomline=false,
	linecolor=red, backgroundcolor=red!2,
},
qed=\qedsymbol
]{thmproofbox}
\declaretheorem[style=thmproofbox, name=]{replacementproof}
\renewenvironment{proof}[1][\proofname]{\vspace{-10pt}\begin{replacementproof}}{\end{replacementproof}}


\definecolor{codegreen}{rgb}{0,0.6,0}
\definecolor{codegray}{rgb}{0.5,0.5,0.5}
\definecolor{codepurple}{rgb}{0.58,0,0.82}
\definecolor{backcolour}{rgb}{0.95,0.95,0.92}

\lstdefinestyle{mystyle}{
	backgroundcolor=\color{backcolour},   
	commentstyle=\color{codegreen},
	keywordstyle=\color{magenta},
	numberstyle=\tiny\color{codegray},
	stringstyle=\color{codepurple},
	basicstyle=\ttfamily\footnotesize,
	breakatwhitespace=false,         
	breaklines=true,                 
	captionpos=b,                    
	keepspaces=true,                 
	numbers=left,                    
	numbersep=5pt,                  
	showspaces=false,                
	showstringspaces=false,
	showtabs=false,                  
	tabsize=2
}

\lstset{style=mystyle}
\title{Matrix Multiplication}

\begin{document}
\begin{titlepage}
	\begin{center}
		\huge Project Work \vfil
		Matrices and their Operations in Python\\

	\end{center}
\end{titlepage}
\tableofcontents
\newpage

\begin{abstract}
	Matrices are a part of Linear Algebra which is used everywhere in Computer Science. It is used in computer graphics to create 2D/3D models, animations, etc. It is used in cryptography (making data secure) by using matrices to store data and a key matrix to encrypt it and its inverse to decrypt it. In this project we will look at operations on matrices in Python.
\end{abstract}
\section{What is a Matrix?}
A matrix is a rectangular array of data\footnote{Data may be of any form, like numbers, expressions or alphabets.} arranged in rows and columns.\\
A matrix looks like the following:
\[A=\begin{bmatrix}
		1 & 10 & 12 \\
		2 & 20 & 22
	\end{bmatrix}_{2\times 3}\]
A matrix is represented by capital letters and the subscript represents $\text{Number of rows}\times \text{Number of columns}$ and the matrix $A$ is called \emph{a 2 by 3 matrix}.\\
In Python, to enter a matrix, we use nested lists like:
\begin{lstlisting}[language=Python, numbers=none]
Matrix=[
	[1,10,12],
	[2,10,22]
]\end{lstlisting}
\subsection{Accessing a Matrix}
A matrix a is generally written as $A=[a_{ij}]_{m\times n}$ where $1\leq i\leq m \land 1\leq j \leq n$. Thus, if we know the location of an element say, \emph{element in row 2 and column 1}, we can write it as $a_{21}$.
In Python as well, if we need to find the \emph{element in row i and column j}, we can return it as:
\begin{lstlisting}[language=Python, numbers=none]
def find_element(matrix, row, column):
	row_index=row-1 #we need to use row-1 as indexes begin from 0
	column_index=column-1
	return matrix[row_index][column_index] \end{lstlisting}
\begin{eg}
	We can also access the elements of a matrix, either row or column wise.
\end{eg}
\begin{explanation}
	To print the matrix row-wise:
	\begin{lstlisting}[language=Python, numbers=none]
def row_wise(matrix):
	for i in range(len(matrix)):
		for j in range(len(matrix[0])):
			print(matrix[i][j],end="\t")
	print("\n") \end{lstlisting}
	If matrix is $A=\begin{bmatrix}
			1 & 10 & 12 \\
			2 & 20 & 22
		\end{bmatrix}$, then, the output is,
	\begin{verbatim}
1       10      12

2       20      22
	\end{verbatim}
	To print the matrix column-wise:
	\begin{lstlisting}[language=Python, numbers=none]
def column_wise(matrix):
	for i in range(len(matrix[0])):
		for j in range(len(matrix)):
			print(matrix[j][i],end="\t")
	print("\n") \end{lstlisting}
	The output in this case is:
	\begin{verbatim}
		1       2
		
		10      20
		
		12      22
	\end{verbatim}
\end{explanation}
\subsection{Null Matrix}
A null matrix is one such that,
\[a_{ij}=0 \quad \forall \ i,j\]
Such a matrix is represented by $O$.
\begin{eg}
	Creating a null matrix of a given order.
\end{eg}
\begin{explanation} \phantom \\
	\begin{lstlisting}[language=Python, numbers=none]
def null(rows,columns):
	null_matrix=[[0 for i in range(columns)] for i in range(rows)] #loop over columns then over rows
	return null_matrix \end{lstlisting}
\end{explanation}

\subsection{Upper and Lower Triangular Matrices}
The upper triangular matrix is a matrix in which all the entries below the diagonal are zero, i.e.,
\[a_{ij}=0\quad \forall \ i\geq j\]
Or,
\[A=\begin{bmatrix}
		a_{11} & a_{12} & \dots  & a_{1n} \\
		0      & a_{22} & \dots  & a_{2n} \\
		\vdots & \vdots & \ddots & \vdots \\
		0      & 0      & \cdots & a_{nn}
	\end{bmatrix}\]
The lower triangular matrix is a matrix in which all the entries above the diagonal are zero\footnote{These definitions are open for discussion. Some authors claim that the matrix must be square, while some do not restrict the matrix. Even though a 'triangle' would not be formed in the case of rectangular matrices, it is acceptable. We have chosen not to restrict the matrices to only square matrices.}, i.e.,
\[a_{ij}=0\quad \forall \ i\leq j\]
Or,
\[A=\begin{bmatrix}
		a_{11} & 0      & \dots  & 0      \\
		a_{21} & a_{22} & \dots  & 0      \\
		\vdots & \vdots & \ddots & 0      \\
		a_{n1} & a_{n2} & \dots  & a_{nn}
	\end{bmatrix}\]
\begin{eg}
	Creating an upper and lower triangular matrix.
\end{eg}
\begin{explanation}
	First for an upper triangular matrix,
	\begin{lstlisting}[language=Python, numbers=none]
def upper_triangular(matrix):
    rows=len(matrix)
    columns=len(matrix[0])
    upper_matrix=null(rows,columns) #null matrix
    for i in range(rows):
        for j in range(columns):
            if i>=j: #Condition for a null matrix
                upper_matrix[i][j]+=matrix[i][j]
            else:
                continue
    return upper_matrix \end{lstlisting}
	If the matrix is $A=\begin{bmatrix}
			1 & 2 & 3 \\
			4 & 5 & 6
		\end{bmatrix}$, then, the output is,
	\begin{verbatim}
		[[1, 0, 0], [4, 5, 0]]
	\end{verbatim}
	Now, for a lower triangular matrix,
	\begin{lstlisting}[language=Python, numbers=none]
def lower_triangular(matrix):
	rows=len(matrix)
	columns=len(matrix[0])
	lower_matrix=null(rows,columns) #null matrix
	for i in range(rows):
		for j in range(columns):
			if j>=i: #Condition for a null
				lower_matrix[i][j]+=matrix[i][j]
			else:
				continue
	return lower_matrix \end{lstlisting}
	The output in this case is,
	\begin{verbatim}
		[[1, 2, 3], [0, 5, 6]]
	\end{verbatim}
\end{explanation}
\subsection{Transpose of a Matrix}
The transpose of a matrix $A=[a_{ij}]_{m\times n}$ is given by,
\[A^T=[a_{ji}]_{n\times m}\]
\begin{eg}
	Create another matrix which is the transpose of a given matrix.
\end{eg}
\begin{explanation}\phantom \\
	\begin{lstlisting}[language=Python, numbers=none]
def transpose(matrix):
	rows=len(matrix)
	columns=len(matrix[0])
	transpose_matrix=null(columns, rows) #null matrix
	for i in range(rows):
		for j in range(columns):
			transpose_matrix[j][i]+=matrix[i][j] #definition of transpose

	return transpose_matrix \end{lstlisting}
	If the matrix is $A=\begin{bmatrix}
			1 & 2 & 3 \\
			4 & 5 & 6 \\
			1 & 1 & 1
		\end{bmatrix}$
\end{explanation}, the output is,
\begin{verbatim}
		[[1, 4, 1], [2, 5, 1], [3, 6, 1]]
	\end{verbatim}
\subsection{Addition of Two Matrices}
Given two matrices $A=[a_{ij}]_{n_1\times m_1}$ and $B=[b_{ij}]_{n_2\times m_2}$, the sum $A+B$ is defined only if $(n_1,m_1)=(n_2,m_2)=(n,m) \ (\text{Say})$, i.e. the matrices have the same order.
\[C=A+B=[a_{ij}+b_{ij}]_{n\times m}\]\
\begin{eg}
	Find the sum of two matrices.
\end{eg}
\begin{explanation}
	It returns the sum of the matrices if it exists, else, returns -1.\\
	\begin{lstlisting}[language=Python, numbers=none]
def Matrix_Sum(matrix1,matrix2):
    if len(matrix1)==len(matrix2) and len(matrix1[0])==len(matrix2[0]): #Checking for compatibility for addition
        rows, columns=len(matrix1), len(matrix1[0])
        sum_matrix=null(rows,columns) #null matrix
        for i in range(rows):
            for j in range(columns):
                sum_matrix[i][j]+=matrix1[i][j]+matrix2[i][j] #definition of sum of matrices
        return sum_matrix
    else:
        return -1 \end{lstlisting}
	If the input matrices are $A=\begin{bmatrix}
			1 & 2 & 3 \\
			4 & 5 & 6 \\
			1 & 1 & 1
		\end{bmatrix}$ and $B=\begin{bmatrix}
			1 & 2 & 3 \\
			4 & 5 & 6 \\
			2 & 2 & 2
		\end{bmatrix}$, the output is,
	\begin{verbatim}
			[[2, 4, 6], [8, 10, 12], [3, 3, 3]]
		\end{verbatim}
	If the matrices are $A=\begin{bmatrix}
			1 & 2 & 3 \\
			4 & 5 & 6 \\
			1 & 1 & 1
		\end{bmatrix}$ and $B=\begin{bmatrix}
			1 & 2 & 3 \\
			4 & 5 & 6 \\
		\end{bmatrix}$, the output is,\\
	-1
\end{explanation}
\subsection{Subtraction of Two Matrices}
Matrices also have am additive inverse, i.e., there exists a matrix $B$ such that,
\[A+B=0\implies B=-A\]
The matrix $B$ is defined as $B=[-a_{ij}]_{n\times m}$ if the matrix $A$ js defined as $A=[a_{ij}]_{n\times m}$.\\
Thus, we can define the operation $A-B$ as $A+(-B)$ which is simple matrix addition.
\begin{eg}
	Find the difference of two matrices.
\end{eg}
\begin{explanation}
	It is similar to the sum of two matrices program.
	\begin{lstlisting}[language=Python, numbers=none]
def Matrix_Difference(matrix1,matrix2):
#The program is matrix1-matrix2, since subtraction is not commutative
matrix2_new=null(len(matrix2),len(matrix2[0]))
for i in range(len(matrix2)):
	for j in range(len(matrix2[0])):
		matrix2_new[i][j]=-matrix2[i][j] #From the definition of the negative of a matrix
return(Matrix_Sum(matrix1,matrix2_new))#From the definition of difference of two matrices
		\end{lstlisting}
	If the input matrices are $A=\begin{bmatrix}
			1 & 2 & 3 \\
			4 & 5 & 6 \\
			1 & 1 & 1
		\end{bmatrix}$ and $B=\begin{bmatrix}
			1 & 2 & 3 \\
			4 & 5 & 6 \\
			2 & 2 & 2
		\end{bmatrix}$, the output is, \begin{verbatim}
			[[0, 0, 0], [0, 0, 0], [-1, -1, -1]]
		\end{verbatim}
\end{explanation}
\section{Division of a Matrix by a Scalar}
When a matrix is divided by a scalar, each element of the matrix is reduced by a factor of the scalar, i.e.,
\[\frac{A}{k}=\left[\frac{a_{ij}}{k}\right]\]
\begin{eg}
	Find the result of a matrix divided by a scalar.
\end{eg}
\begin{explanation}
	Using the above definition,
	\begin{lstlisting}[language=Python, numbers=none]
def Division(mx, scalar: int):
    new_matrix = null(len(mx), len(mx[0]))
    for i in range(len(mx)):
        for j in range(len(mx[0])):
            new_matrix[i][j] += mx[i][j] / scalar
    return new_matrix \end{lstlisting}
\end{explanation}
\subsection{Minor Matrix of an Element}
The minor matrix of an element is the matrix obtained from removing the $i^{\text{th}}$ row and $j^{\text{th}}$ column of a matrix.\
\begin{eg}
	Find the minor matrix of a matrix given the location of the matrix,
\end{eg}
\begin{explanation}
	For this, first we exclude the $i^{\text{th}}$ row. Then, we exclude the $j^{\text{th}}$ column \footnotetext{The following is an incredible one-liner.}.
	\begin{lstlisting}[language=Python, numbers=none]
def Minor_Matrix(mx,r,c):
    return [row[:c]+row[c+1:] for row in (mx[:r]+mx[r+1:])] \end{lstlisting}
\end{explanation}
\subsection{Determinant of a Matrix}
The determinant is a scalar value that is a function of the entries of a square matrix. It is denoted by $det(A)$, $|A|$ or $\Delta$.\\
If $A=\begin{bmatrix}
		a_{11} & a_{12} \\
		a_{21} & a_{22}
	\end{bmatrix}$ then, $\Delta =a_{11}a_{22}-a_{21}a_{12}$.\\
If $A=\begin{bmatrix}
		a_{11} & a_{12} & a_{13} \\
		a_{21} & a_{22} & a_{23} \\
		a_{31} & a_{32} & a_{33}
	\end{bmatrix}$ then, $\Delta=a_{11}\begin{vmatrix}
		a_{22} & a_{23} \\
		a_{32} & a_{33}
	\end{vmatrix}+a_{22}\begin{vmatrix}
		a_{21} & a_{23} \\
		a_{31} & a_{33}
	\end{vmatrix}+a_{33}\begin{vmatrix}
		a_{21} & a_{22} \\
		a_{31} & a_{32}
	\end{vmatrix}$\\
Before finding the determinant of a matrix, let us look at Cofactors.
\subsubsection{Cofactors}
The cofactor of an element $a_{ij}$ is denoted by $A_{ij}$ and is given by,
\[A_{ij}=(-1)^{i+j}M_{ij}\]
Where, $M_{ij}$ is the determinant of the minor matrix of the element $a_{ij}$.
The determinant of a matrix is thus given by,
\[\Delta=\sum\limits_{j=0}^n a_{1j}A_{1j}\]
Thus, the determinant of a matrix is the sum of the elements multiplied with their cofactors.
\begin{eg}
	Find the determinant of a matrix.
\end{eg}
\begin{explanation}
	Since we have a closed form for the determinant of a $2\times 2$ matrix, we can reduce all determinants\footnote{So, we need to use recursion.} to finding the determinant of a smaller matrix by using cofactors.
	\begin{lstlisting}[language=Python, numbers=none]
def Determinant(mx):
	# for the base case when the matrix is 2*2
	if len(mx) == 2:
		return mx[0][0] * mx[1][1] - mx[0][1] * mx[1][0]
	determinant = 0
	# We will only expand along the first row
	for i in range(len(mx)):
		determinant += (
			((-1) ** i) * mx[0][i] * Determinant(Minor_Matrix(mx, 0, i))
		)  # By definition
	return determinant \end{lstlisting}
\end{explanation}
\section{Inverse of a Matrix}
If $A$ is a square matrix of order $m$, and there exists another square matrix $B$ of the same order $m$, such that $AB=BA=I$, then $B$ is called the inverse matrix of $A$ and it is denoted by $A^{-1}$. In that case $A$ is said to be invertible. The inverse of a matrix if it exists is unique.\
\subsection{Adjoint of a Matrix}
The adjoint of a square matrix $A=[a_{ij}]_{n\cross n}$ is defined as the transpose of the matrix $[A_{ij}]_{n\cross n}$, i.e.,
\[ \text{adj} \ A=[A_{ij}]_{n\cross n}\]
\begin{eg}
	Find the adjoint of a matrix.
\end{eg}
\begin{explanation} \phantom \\
	\begin{lstlisting}[language=Python, numbers=none]
def Adjoint(mx):
    temp_matrix = []  # Empty matrix before the transpose
    for i in range(len(mx)):
        row = []  # Initialise each row
        for j in range(len(mx)):
            row.append(
                ((-1) ** (i + j)) * Determinant(Minor_Matrix(mx, i, j))
            )  # from the definition of adjoint
        temp_matrix.append(row)
    return transpose(temp_matrix) \end{lstlisting}
\end{explanation}
The inverse of a matrix can be calculated by using:
\[A^{-1}=\frac{1}{|A|}\text{adj} \ A\]
\begin{eg}
	Find the inverse of a matrix.
\end{eg}
\begin{explanation}
	Using the above definition,
	\begin{lstlisting}[language=Python, numbers=none]
def Inverse(mx):
	if Determinant(mx) == 0:
		return None  # The inverse does not exist in this case
	else:
		return Division(Adjoint(mx), Determinant(mx))  # By definition \end{lstlisting}
\end{explanation}
\section{Multiplication of Two Matrices}
The product of two matrices is defined if the number of columns of $A$ is equal to the number of rows of $B$. If $A=[a_{ij}]_{m\cross n}$ and $B=[b_{jk}]_{n\cross p}$, then the $i^{th}$ row of $A$ is $ \begin{bmatrix}
		a_{i1} & a_{i2} & \cdots & a_{in}
	\end{bmatrix}$ and the $k^{th}$ column of $B$ is $ \begin{bmatrix}
		b_{1k} \\
		b_{2k} \\
		\vdots \\
		b_{nk}
	\end{bmatrix}$, then $c_{ik}=\sum\limits_{j=1}^{n}a_{ij}b_{jk}$. The matrix $C=[c_{ik}]_{m\cross p}$ is the product $A B$. Matrix Multiplication is not commutative but is associative and distributive. Moreover, $AB=0 \nRightarrow A=0 \lor B=0$.\\
Here, we will take two approaches to find the product of two matrices.
\begin{eg}
	Find the product of two matrices.
\end{eg}
\begin{explanation}
	Here, we will use the exact definition used above. We will loop 3 times.
	\begin{lstlisting}[language=Python, numbers=none]
def Multiplication(mx1, mx2):
    if len(mx1[0]) != len(mx2):
        return "The matrices are incompatible for multiplication."
    else:
        result = null(len(mx1), len(mx2[0]))
        for i in range(len(mx1)):  # Loop through rows of first matrix
            for j in range(len(mx2[0])):  # Loop through each column of second matrix
                for k in range(len(mx2)):  # Loop through each row of second matrix
                    result[i][j] += mx1[i][k] * mx2[k][j]  # By definition
        return result \end{lstlisting}
	The following can also be done one line as follows:
	\begin{lstlisting}[language=Python, numbers=none]
def Matrix_Multiplication(mx1,mx2):
    return [[sum([mx1[i][k]*mx2[k][j] for k in range(len(mx1[0]))]) for j in range(len(mx2[0]))] for i in range(len(mx1))] \end{lstlisting}
	This method is however very slow, as it loops three times. If the matrices are square matrices of the same order (n), then the program takes $O(n*n*n)=O(n^3)$ time.
\end{explanation}
\subsection{Strassen Algorithm}
Strassen Algorithm\footnote{named after the German mathematician Volker Strassen.} is an algorithm for multiplication of two matrices. It is much faster than the traditional multiplication algorithms (like the one mentioned above). However, it only works for square matrices of the order $2^n$, where $n\in \mathrm{N}$.
\subsubsection{Joining Matrices}
\begin{eg}
	Join any two matrices horizontally.
\end{eg}
\begin{explanation}
	\phantom \\
	\begin{lstlisting}[language=Python, numbers=none]
def joining_horizontally(a: list, b: list) -> list[list]:
	n = len(a)
	new_matrix = null(
		len(a), len(a[0]) + len(b[0])
	)  # Null matrix of the same order as the expected output
	for i in range(n):
		for j in range(n):  
			new_matrix[i][j] = a[i][j]  # First matrix
			new_matrix[i][j + n] = b[i][j]  # Second matrix
	return new_matrix \end{lstlisting}
	If the input matrices are $A=\begin{bmatrix}
			1 & 2 & 3 \\
			4 & 5 & 6 \\
			7 & 8 & 9
		\end{bmatrix}$ and $B=\begin{bmatrix}
			10 & 11 & 12 \\
			13 & 14 & 15 \\
			16 & 17 & 18
		\end{bmatrix}$, then the output matrix is, $\begin{bmatrix}
			1 & 2 & 3 & 10 & 11 & 12 \\
			4 & 5 & 6 & 13 & 14 & 15 \\
			7 & 8 & 9 & 16 & 17 & 18
		\end{bmatrix}$.
\end{explanation}
\begin{eg}
	Join any two matrices vertically.
\end{eg}
\begin{explanation}
	\phantom \\
	\begin{lstlisting}[language=Python, numbers=none]
def joining_vertically(a: list[list], b: list[list]) -> list[list]:
    n = len(mx1)
    new_matrix = []  # Empty list
    for i in mx1:
        new_matrix.append(i)  # First the first matrix
    for j in mx2:
        new_matrix.append(j)  # Now the second matrix
    return new_matrix \end{lstlisting}
	If the input matrices are $A=\begin{bmatrix}
			1 & 2 & 3 \\
			4 & 5 & 6 \\
			7 & 8 & 9
		\end{bmatrix}$ and $B=\begin{bmatrix}
			10 & 11 & 12 \\
			13 & 14 & 15 \\
			16 & 17 & 18
		\end{bmatrix}$, then the output matrix is, $\begin{bmatrix}
			1  & 2  & 3  \\
			4  & 5  & 6  \\
			7  & 8  & 9  \\
			10 & 11 & 12 \\
			13 & 14 & 15 \\
			16 & 17 & 18
		\end{bmatrix}$.
\end{explanation}
\subsection{Splitting a Matrix into 4 Smaller Matrices}
\begin{eg}
	Split a given matrix into 4 smaller matrices, given the order of the matrix is of the form $2^n\times 2^n$.
\end{eg}
\begin{explanation}
	We use simple list slicing to achieve the desired output.
	\begin{lstlisting}[language=Python, numbers=none]
def split(mx: list[list]) -> list[list]:
	if len(mx) == len(mx[0]) and pow_2(len(mx)):
		n = len(mx) // 2
		a = mx[:n]
		b = mx[n:]
		a_11 = [a[i][:n] for i in range(n)]
		a_12 = [a[i][n:] for i in range(n)]
		a_13 = [b[i][:n] for i in range(n)]
		a_14 = [b[i][n:] for i in range(n)]
	return a_11, a_12, a_13, a_14 \end{lstlisting}
\end{explanation}
\begin{theorem}
	Product of two $2\times 2$ matrices.
\end{theorem}
\begin{proof}
	Let the matrices be:\\
	\[\begin{bmatrix}a_{11}&a_{12}\\ a_{21}&a_{22}\end{bmatrix} \ \land \ \begin{bmatrix} b_{11}&b_{12}\\b_{21}&b_{22}\end{bmatrix}\]
	Then the product matrix is given, by,
	\[\begin{bmatrix} C_{11}& C_{12} \\ C_{21} & C_{22} \end{bmatrix}\]
	Where, the elements are,
	\[\begin{split}
			C_{11}&=a_{11}b_{11}+a_{12}b_{21} \\
			C_{12}&=a_{11}b_{12}+a_{12}b_{22}\\
			C_{21}&=a_{21}b_{11}+a_{22}b_{21}\\
			C_{22}&=a_{21}b_{12}+a_{22}b_{22}
		\end{split}\]
	However, this uses 8 multiplications. Strassen found a way to use only 7 multiplications. This is a bit faster as additions are faster than multiplications.
	For this, we calculate the following 7 products:
	\[\begin{split}
			P_1&=(a_{11}+a_{22})(b_{11}+b_{22})\\
			P_2&=a_{22}(b_{21}-b_{11})\\
			P_3&=(a_{11}+a_{12})b_{22}\\
			P_4&=(a_{12}-a_{22})(b_{21}+b_{22})\\
			P_5&=a_{11}(b_{12}-b_{22})\\
			P_6&=(a_{21}+a_{22})b_{11}\\
			P_7&=(a_{11}-a_{21})(b_{11}+b_{12})
		\end{split}\]
	Now, calculating the elements,
	\[\begin{split}
			C_{11}&=P_1+P_4-P_5-P_7\\
			C_{12}&=P_3+P_5\\
			C_{21}&=P_2+P_6\\
			C_{22}&=P_1+P_5-P_6-P_7
		\end{split}\]
	Now, for a bigger matrix, i.e. of the order $n$, the elements $a_{ij}$ will be matrices and the product will be calculated recursively.\\
	For the time complexity of the previous method,

	We reduce the order of the matrix to $\frac{n}{2}$ in each step and calculate 8 products and we add $n^2$ times, thus,
	\[T(n)= 8T(\frac{n}{2})+n^2\]
	On solving the recurrence relation, we get,
	\[T(n)=O(n^3)\]
	Which is the same as using 3 loops.\\
	Now, for the time complexity of Strassen Algorithm. Similarly,
	\[T(n)=7T(\frac{n}{2})+n^2\]
	Here, we get,
	\[T(n)=O(n^{\log_2 7})=O(2^{2.81})\]
	Which is clearly faster than the above mentioned methods!
\end{proof}
\begin{eg}
	Using the Strassen Algorithm.
\end{eg}
\begin{explanation}
	It works on the principle of divide and conquer.
	\begin{lstlisting}[language=Python, numbers=none]
def strassen(mx1: list[list], mx2: list[list]) -> list[list]:
	if len(mx1) == 1:
		return [[mx1[0][0] * mx2[0][0]]]
	a_11, a_12, a_21, a_22 = split(mx1)
	b_11, b_12, b_21, b_22 = split(mx2)
	p_1, p_2, p_3, p_4, p_5, p_6, p_7 = (
		strassen(Matrix_Sum(a_11, a_22), Matrix_Sum(b_11, b_22)),
		strassen(a_22, Matrix_Difference(b_21, b_11)),
		strassen(Matrix_Sum(a_11, a_12), b_22),
		strassen(Matrix_Difference(a_12, a_22), Matrix_Sum(b_21, b_22)),
		strassen(a_11, Matrix_Difference(b_12, b_22)),
		strassen(Matrix_Sum(a_21, a_22), b_11),
		strassen(Matrix_Difference(a_11, a_21), Matrix_Sum(b_11, b_12)),
	)  # Exactly from the theory
	c_11, c_12, c_21, c_22 = (
		Matrix_Difference(Matrix_Sum(Matrix_Sum(p_1, p_2), p_4), p_3),
		Matrix_Sum(p_3, p_5),
		Matrix_Sum(p_2, p_6),
		Matrix_Difference(Matrix_Sum(p_1, p_5), Matrix_Sum(p_6, p_7)),
	)  # Exactly from the theory
	return joining_vertically(
		joining_horizontally(c_11, c_12), joining_horizontally(c_21, c_22)
	) \end{lstlisting}
\end{explanation}

\section{Binary Files}

\end{document}