\documentclass[titlepage]{article}
\usepackage[margin=0.5in]{geometry}
\usepackage{amsmath}
\usepackage{amsfonts}
\usepackage{amssymb}
\usepackage{amsthm}
\usepackage{graphicx}
\usepackage{physics}
\usepackage{tikz}
\usepackage{pgfplots}
\pgfplotsset{compat=1.15}
\usepackage{mathrsfs}
\usetikzlibrary{arrows,patterns, backgrounds, calc,fadings,shadows.blur, shapes}
\theoremstyle{plain}
\usepackage[most]{tcolorbox}
\usepackage{physunits}
\usepackage{float}
\usepackage{xcolor}
\usepackage{microtype}
\usepackage{thmtools}
\usepackage[framemethod=TikZ]{mdframed}
\usepackage{steinmetz}
\usepackage{adjustbox}
\usepackage{listings}
\mdfsetup{skipabove=1em,skipbelow=0em}
\tcbuselibrary{xparse}
\usepackage[font=small,labelfont=bf,margin=\parindent,tableposition=top]{caption}
\newenvironment{solution}
{\renewcommand\qedsymbol{$\blacksquare$}\begin{proof}[Solution]}
	{\end{proof}}


\declaretheoremstyle[
headfont=\bfseries\sffamily\color{blue!70!black}, bodyfont=\normalfont,
mdframed={
	linewidth=1pt,
	rightline=false, topline=false, bottomline=false,
	linecolor=blue, backgroundcolor=blue!5,
}
]{thmbluebox}
\declaretheorem[style=thmbluebox, numbered=no, name=Problem]{eg}


\declaretheoremstyle[
headfont=\bfseries\sffamily\color{blue!70!black}, bodyfont=\normalfont,
numbered=no,
mdframed={
	linewidth=1pt,
	rightline=false, topline=false, bottomline=false,
	linecolor=blue, backgroundcolor=blue!1,
},
]{thmexplanationbox}
\declaretheorem[style=thmexplanationbox, name=Code]{tmpexplanation}
\newenvironment{explanation}[1][]{\vspace{-10pt}\begin{tmpexplanation}}{\end{tmpexplanation}}


\declaretheoremstyle[
headfont=\bfseries\sffamily\color{red!70!black}, bodyfont=\normalfont,
mdframed={
	linewidth=1pt,
	rightline=false, topline=false, bottomline=false,
	linecolor=red, backgroundcolor=red!5,
}
]{thmredbox}
\declaretheorem[style=thmredbox,numbered=no, name=\phantom{.}]{theorem}

\declaretheoremstyle[
headfont=\bfseries\sffamily\color{red!70!black}, bodyfont=\normalfont,
numbered=no,
mdframed={
	linewidth=1pt,
	rightline=false, topline=false, bottomline=false,
	linecolor=red, backgroundcolor=red!2,
},
qed=\qedsymbol
]{thmproofbox}
\declaretheorem[style=thmproofbox, name=]{replacementproof}
\renewenvironment{proof}[1][\proofname]{\vspace{-10pt}\begin{replacementproof}}{\end{replacementproof}}


\definecolor{codegreen}{rgb}{0,0.6,0}
\definecolor{codegray}{rgb}{0.5,0.5,0.5}
\definecolor{codepurple}{rgb}{0.58,0,0.82}
\definecolor{backcolour}{rgb}{0.95,0.95,0.92}

\lstdefinestyle{mystyle}{
	backgroundcolor=\color{backcolour},   
	commentstyle=\color{codegreen},
	keywordstyle=\color{magenta},
	numberstyle=\tiny\color{codegray},
	stringstyle=\color{codepurple},
	basicstyle=\ttfamily\footnotesize,
	breakatwhitespace=false,         
	breaklines=true,                 
	captionpos=b,                    
	keepspaces=true,                 
	numbers=left,                    
	numbersep=5pt,                  
	showspaces=false,                
	showstringspaces=false,
	showtabs=false,                  
	tabsize=2
}

\lstset{style=mystyle}
\title{Matrix Multiplication}

\begin{document}
	\begin{titlepage}
		\begin{center}
			\huge Project Work \vfil
			Matrices and their Operations in Python\\
			
		\end{center}
	\end{titlepage}
	\tableofcontents
	\newpage
	
	\begin{abstract}
		Matrices are a part of Linear Algebra which is used everywhere in Computer Science. It is used in computer graphics to create 2D/3D models, animations, etc. It is used in cryptography (making data secure) by using matrices to store data and a key matrix to encrypt it and its inverse to decrypt it. In this project we will look at operations on matrices in Python.
	\end{abstract}
	\section{What is a Matrix?}
	A matrix is a rectangular array of data\footnote{Data may be of any form, like numbers, expressions or alphabets.} arranged in rows and columns.\\
	A matrix looks like the following:
	\[A=\begin{bmatrix}
		1 & 10 & 12 \\
		2&20& 22
	\end{bmatrix}_{2\times 3}\]
	A matrix is represented by capital letters and the subscript represents $\text{Number of rows}\times \text{Number of columns}$ and the matrix $A$ is called \emph{a 2 by 3 matrix}.\\
	In Python, to enter a matrix, we use nested lists like: 
	\begin{lstlisting}[language=Python]
Matrix=[
	[1,10,12],
	[2,10,22]
]\end{lstlisting}
	\subsection{Accessing a Matrix}
	A matrix a is generally written as $A=[a_{ij}]_{m\times n}$ where $1\leq i\leq m \land 1\leq j \leq n$. Thus, if we know the location of an element say, \emph{element in row 2 and column 1}, we can write it as $a_{21}$.
	In Python as well, if we need to find the \emph{element in row i and column j}, we can return it as:
	\begin{lstlisting}[language=Python]
def find_element(matrix, row, column):
	row_index=row-1 #we need to use row-1 as indexes begin from 0
	column_index=column-1
	return matrix[row_index][column_index] \end{lstlisting}	
	\begin{eg}
		We can also access the elements of a matrix, either row or column wise.
	\end{eg}
	\begin{explanation}
		To print the matrix row-wise:
		\begin{lstlisting}[language=Python]
def row_wise(matrix):
	for i in range(len(matrix)):
		for j in range(len(matrix[0])):
			print(matrix[i][j],end="\t")
	print("\n") \end{lstlisting}
	If matrix is $A=\begin{bmatrix}
		1 & 10 & 12 \\
		2&20& 22
	\end{bmatrix}$, then, the output is,
	\begin{verbatim}
1       10      12

2       20      22
	\end{verbatim}
		To print the matrix column-wise:
	\begin{lstlisting}[language=Python]
def column_wise(matrix):
	for i in range(len(matrix[0])):
		for j in range(len(matrix)):
			print(matrix[j][i],end="\t")
	print("\n") \end{lstlisting}
	The output in this case is:
	\begin{verbatim}
		1       2
		
		10      20
		
		12      22
	\end{verbatim}
	\end{explanation}
	\subsection{Null Matrix}
	A null matrix is one such that,
	\[a_{ij}=0 \quad \forall \ i,j\]
	Such a matrix is represented by $O$.
	\begin{eg}
		Creating a null matrix of a given order.
	\end{eg}
	\begin{explanation} \phantom \\
		\begin{lstlisting}[language=Python]
def null(rows,columns):
	null_matrix=[[0 for i in range(columns)] for i in range(rows)] #loop over columns then over rows
	return null_matrix \end{lstlisting}
	\end{explanation}

	\subsection{Upper and Lower Triangular Matrices}
	The upper triangular matrix is a matrix in which all the entries below the diagonal are zero, i.e.,
	\[a_{ij}=0\quad \forall \ i\geq j\]
	Or,
	\[A=\begin{bmatrix}
		a_{11} & a_{12} & \dots & a_{1n}\\
		0 & a_{22} & \dots & a_{2n}\\
		\vdots & \vdots & \ddots & \vdots \\
		0& 0& \cdots & a_{nn}
	\end{bmatrix}\]
	The lower triangular matrix is a matrix in which all the entries above the diagonal are zero\footnote{These definitions are open for discussion. Some authors claim that the matrix must be square, while some do not restrict the matrix. Even though a 'triangle' would not be formed in the case of rectangular matrices, it is acceptable. We have chosen not to restrict the matrices to only square matrices.}, i.e.,
	\[a_{ij}=0\quad \forall \ i\leq j\]
	Or,
	\[A=\begin{bmatrix}
		a_{11}& 0 & \dots & 0\\
		a_{21}& a_{22} & \dots & 0\\
		\vdots & \vdots & \ddots & 0\\
		a_{n1}& a_{n2}& \dots & a_{nn}
	\end{bmatrix}\]
	\begin{eg}
		Creating an upper and lower triangular matrix.
	\end{eg}
	\begin{explanation}
		First for an upper triangular matrix,
		\begin{lstlisting}[language=Python]
def upper_triangular(matrix):
    rows=len(matrix)
    columns=len(matrix[0])
    upper_matrix=null(rows,columns) #null matrix
    for i in range(rows):
        for j in range(columns):
            if i>=j: #Condition for a null matrix
                upper_matrix[i][j]+=matrix[i][j]
            else:
                continue
    return upper_matrix \end{lstlisting}
	If the matrix is $A=\begin{bmatrix}
		1 & 2& 3\\
		4& 5& 6
	\end{bmatrix}$, then, the output is,
	\begin{verbatim}
		[[1, 0, 0], [4, 5, 0]]
	\end{verbatim}
	Now, for a lower triangular matrix,
	\begin{lstlisting}[language=Python]
def lower_triangular(matrix):
	rows=len(matrix)
	columns=len(matrix[0])
	lower_matrix=null(rows,columns) #null matrix
	for i in range(rows):
		for j in range(columns):
			if j>=i: #Condition for a null
				lower_matrix[i][j]+=matrix[i][j]
			else:
				continue
	return lower_matrix \end{lstlisting}
	The output in this case is,
	\begin{verbatim}
		[[1, 2, 3], [0, 5, 6]]
	\end{verbatim}
	\end{explanation}
	\subsection{Transpose of a Matrix}
	The transpose of a matrix $A=[a_{ij}]_{m\times n}$ is given by,
	\[A^T=[a_{ji}]_{n\times m}\]
	\begin{eg}
		Create another matrix which is the transpose of a given matrix.
	\end{eg}
	\begin{explanation}
		\begin{lstlisting}[language=Python]
def transpose(matrix):
	rows=len(matrix)
	columns=len(matrix[0])
	transpose_matrix=null(columns, rows) #null matrix
	for i in range(rows):
		for j in range(columns):
			transpose_matrix[j][i]+=matrix[i][j] #definition of transpose

	return transpose_matrix \end{lstlisting}
	If the matrix is $A=\begin{bmatrix}
		1&2&3\\
		4&5&6\\
		1&1&1
	\end{bmatrix}$
	\end{explanation}, the output is,
	\begin{verbatim}
		[[1, 4, 1], [2, 5, 1], [3, 6, 1]]
	\end{verbatim}
\end{document}